\documentclass{ctexart}

\usepackage{CJKfntef}%提供对汉字进行标记的功能的宏包,比如在汉字上面添加下划线、下波浪线和删除线等
\usepackage{listings}
\usepackage{fontspec}
\usepackage{graphicx}
\usepackage{xcolor}
\usepackage{slashbox}
\usepackage{amsmath}
\usepackage{setspace}%使用间距宏包

\DeclareGraphicsExtensions{.eps,.ps,.jpg,.bmp}
\newfontfamily\menlo{Monaco}
\definecolor{mygreen}{rgb}{0,0.6,0}
\definecolor{mygray}{rgb}{0.5,0.5,0.5}
\definecolor{mymauve}{rgb}{0.58,0,0.82}

\lstset{
    numbers = left,
    language = python,
    %numberstyle = \tiny,
    numberstyle=\menlo,
    basicstyle = \footnotesize\menlo,
    showspaces = false, % 显示空格
    showstringspaces = false,  % 字符串中显示空格
    breaklines = true,                 % automatic line breaking only at whitespace
    frame = single,
    keywordstyle = \color{blue},
    commentstyle=\color{mygreen},    % comment style
    stringstyle=\color{mymauve},     % string literal style
    rulesepcolor = \color{ red!20!green!20!blue!20},
    %commentstyle = \color{red!50!green!50!blue!50},
    escapeinside={\%*}{*)}          % if you want to add LaTeX within your code
}

\title{\bf 防空导弹主级燃料相对质量因数计算}
\author{班号:02021302 学号:2013300052 姓名:缪宇驰}
\date{\today}

\begin{document}
\maketitle
\tableofcontents

\clearpage
\section{设计要求}
1.根据已知条件,采用数值积分法求解相对量运动微分方程组,计算燃料相对质量因数。

2.综合运用积分、插值计算等计算方法,采用C、C++,或者Matlab等语言的一种,编制计算程序。

%\subsection{平时训练}
%\subsection{暑假集训}
%\subsection{区域赛}

\section{已知条件}
1.分离条件:速度$V_0$=500m/s,时间$t_0$=3s,x方向位置$x_0$=674m,y方向位置$y_0$=329m,初始攻角$\alpha$=$1.5^{\circ}$,初始弹道倾角$\theta$=$26^{\circ}$。
~\\

2.气动参数

阻力因数$C_x(Ma,\alpha)$ 插值表:

\begin{tabular}{|c|c|c|c|c|c|}
\hline
\backslashbox[2cm]{Ma}{$\alpha$} & 2 & 4 & 6 & 8 & 10\\
\hline
1.5 & 0.0430 & 0.0511 & 0.0651 & 0.0847 & 0.1120\\
\hline
2.1 & 0.0360 & 0.0436 & 0.0558 & 0.0736 & 0.0973\\
\hline
2.7 & 0.0308 & 0.0372 & 0.0481 & 0.0641 & 0.0849\\
\hline
3.3 & 0.0265 & 0.0323 & 0.0419 & 0.0560 & 0.0746\\
\hline
4.0 & 0.0222 & 0.0272 & 0.0356 & 0.0478 & 0.0644\\
\hline
\end{tabular}

~\\

升力因数斜率$C^{\alpha}_y(Ma,\alpha)$ 插值表:

\begin{tabular}{|c|c|c|c|c|c|c|}
\hline
\backslashbox[2cm]{Ma}{$\alpha$} & 1 & 2 & 4 & 6 & 8 & 10\\
\hline
1.5 & 0.0302 & 0.0304 & 0.0306 & 0.0309 & 0.0311 & 0.0313\\
\hline
2.0 & 0.0279 & 0.0280 & 0.0284 & 0.0286 & 0.0288 & 0.0290\\
\hline
2.5 & 0.0261 & 0.0264 & 0.0267 & 0.0269 & 0.0272 & 0.0274\\
\hline
3.0 & 0.0247 & 0.0248 & 0.0251 & 0.0254 & 0.0257 & 0.0259\\
\hline
3.5 & 0.0226 & 0.0227 & 0.0231 & 0.0233 & 0.0236 & 0.0238\\
\hline
4.0 & 0.0209 & 0.0210 & 0.0213 & 0.0216 & 0.0219 & 0.0221\\
\hline
\end{tabular}

注:以上攻角$\alpha$单位均为度
~\\

3.发动机参数:比冲Is=2156N$\cdot$s/kg;重力加速度g=9.801m/$s^2$;推重比$\overline{P}$=2.2;翼载$p_0$=5880N/$m^2$。

4.导引规律:三点法。目标匀速直线等高迎头飞行,$V_T$=420m/s,$y_T$=15000m,$D_{T0}$=34200m(弹目斜距)。

5.大气模型:参见教科书。

\section{题目求解}
题目限定导引律是三点法,所以先给出三点法导引时相对量运动微分方程组(当$\alpha$不为常数时):

\begin{align}
\dfrac{dv}{d\mu} &= \dfrac{I_s}{1-\mu} - \dfrac{{\rho}v^2C_xI_s}{2\overline{P}p_0(1-\mu)} - \dfrac{I_ssin\theta}{\overline{P}} \\
\dfrac{d\theta}{d\mu} &= \dfrac{\dfrac{v_T}{y_T}[\dfrac{I_s}{\overline{P}g} + \dfrac{1}{v^2sin\theta}(v\dfrac{dy}{d\mu}-y\dfrac{dv}{d\mu})]}{1+({\cot}q_0-\dfrac{{\mu}I_sv_T}{\overline{P}gy_T})\cot\theta} \\
\dfrac{dy}{d\mu} &= \dfrac{I_s}{\overline{P}g}vsin\theta \\
\dfrac{dx}{d\mu} &= \dfrac{I_s}{\overline{P}g}vcos\theta \\
\alpha &= \dfrac{v\dfrac{d\theta}{d\mu}+\dfrac{I_scos\theta}{\overline{P}}}{\dfrac{{\rho}v^2C^{\alpha}_{yTR}I_s}{2\overline{P}p_0(1-\mu)}+\dfrac{I_s}{1-\mu}} \\
\mu &= \dfrac{\overline{P}gy_T}{I_sv_T}(cotq_0-cotq) \\
D_r &= \sqrt{x^2+y^2} \\
Ma &= \dfrac{v}{c}
\end{align}

这里使用四阶Runge-Kutta法来求解上面的相对量微分方程,Runge-Kutta法计算公式为:

\begin{equation}
\left\{
\begin{aligned}
K_1 &= f(x_n,y_n) \\
K_2 &= f(x_n+\frac{h}{2},y_n+\frac{h}{2}K_1) \\
K_3 &= f(x_n+\frac{h}{2},y_n+\frac{h}{2}K_2) \\
K_4 &= f(x_n+h,y_n+hK_3) \\
y_{n+1} &= y_n + \frac{h}{6}(K_1+2K_2+2K_3+K_4)
\end{aligned}
\right.
\end{equation}


先初始化与导弹和发动机相关的参数,比如速度、位置、攻角、弹道倾角、比冲、翼载等。然后取步长h=0.001进行计算,使用龙格库塔法先求解公式(1),得到$v_{n+1}$的值,然后将求解出的$v_{n+1}$代入到公式(3)、(4)中解出$y_{n+1}$和$x_{n+1}$。接着使用龙格库塔法求解公式(2),其中将公式中的$dy$/$d\mu$和$dv$/$d\mu$由公式(3)和公式(1)的结果代换,得到$\theta_{n+1}$。然后将$d\theta$/$d\mu$和计算出来的其他值代入到公式(5)中求解出$\alpha_{n+1}$,而$\mu_{n+1}$=$\mu_n$+h。迭代上述步骤直到导弹的高度$y_n$大于目标的高度$y_t$。\textbf{注意计算出来的马赫数$Ma$和攻角$\alpha$要作为参数进行二维插值得到当前飞行状态的阻力因数$C_{xn}$和升力因数斜率$C^{\alpha}_{yn}$,飞行开始阶段的攻角和马赫数不在插值表中,此时需要使用外插法设定当前参数,其他情况使用内插法。}而当前飞行高度的密度$\rho$和音速$c$需要对大气模型进行一维插值得到。这里使用美国标准大气模型(1976 Standard Atmosphere Calculator[0-1000 km])作为数据来源,考虑求解问题的特殊性,编程时只用高度为0-26km,步长为1km 的大气密度和音速数据。\textbf{最后计算得到导弹主级燃料相对质量因数$\mu=0.333$。击中目标时的导弹攻角$\alpha=5.331^{\circ}$}

\clearpage

\section{程序代码}
使用python库numpy存储数据,数值计算库scipy进行插值
\begin{lstlisting}
#!/usr/bin/python
#filename       :main.py
#author         :Yuchi Miao
#created time	:2016/4/18 19:44:10
#last modified  :2016/4/19 15:58:10
#python_version	:2.7.6
#version        :1.0
#description	:use the numerical integration method(Runge_Kutta Method) to
#                solve the first order differential equation of continuous
#                system
#notes          :need to set the value in numerical extrapolate manually
#===========================================================================

import math
import numpy as np
import scipy.interpolate as itp
import matplotlib.pyplot as plt

v0 = 500
x0 = 674
y0 = 329
alpha0_ang = 1.5
theta0_ang = 26
Is = 2156
g = 9.801
P = 2.2
p0 = 5880
#t0 = 3

#======Machine Parameter======
vt = 420
yt0 = 15000
Dt0 = 34200
xt0 = math.sqrt(Dt0 ** 2 - (yt0 - y0) ** 2) + x0;

#=======Drag Coefficient======
alpha_tab1 = np.linspace(2, 10, 5, endpoint=True)
ma_tab1 = np.array([1.5, 2.1, 2.7, 3.3, 4.0])
cx_tab = np.array([[0.0430, 0.0511, 0.0651, 0.0847, 0.1120],
                   [0.0360, 0.0436, 0.0558, 0.0736, 0.0973],
                   [0.0308, 0.0372, 0.0481, 0.0641, 0.0849],
                   [0.0265, 0.0323, 0.0419, 0.0560, 0.0746],
                   [0.0222, 0.0272, 0.0356, 0.0478, 0.0644]])

#======Lift Coefficient======
alpha_tab2 = np.array([1, 2, 4, 6, 8, 10])
ma_tab2 = np.linspace(1.5, 4.0, 6, endpoint=True)
cy_tab = np.array([[0.0302, 0.0304, 0.0306, 0.0309, 0.0311, 0.0313],
                   [0.0279, 0.0280, 0.0284, 0.0286, 0.0288, 0.0290],
                   [0.0261, 0.0264, 0.0267, 0.0269, 0.0272, 0.0274],
                   [0.0247, 0.0248, 0.0251, 0.0254, 0.0257, 0.0259],
                   [0.0226, 0.0227, 0.0231, 0.0233, 0.0236, 0.0238],
                   [0.0209, 0.0210, 0.0213, 0.0216, 0.0219, 0.0221]])


#======Atmospheric Model(US1976)===
high_tab = np.linspace(0, 26000, 27, endpoint=True)
rho_tab = np.loadtxt("rho.txt")
speed_tab = np.loadtxt("speed.txt")

u = np.zeros(666, np.float)
v = np.zeros(666, np.float)
x = np.zeros(666, np.float)
y = np.zeros(666, np.float)
alpha_ang = np.zeros(666, np.float)
alpha_rad = np.zeros(666, np.float)
theta_ang = np.zeros(666, np.float)
theta_rad = np.zeros(666, np.float)
cx = np.zeros(666, np.float)
cy = np.zeros(666, np.float)
#xt = np.zeros(666, np.float)
i = 1
u[i] = 0
#xt[i] = xt0
x[i] = x0
y[i] = y0
v[i] = v0
cotq0 = xt0 / yt0

c = itp.interp1d(high_tab, speed_tab)(y[i])
ma = v[i] / c
alpha_ang[i] = alpha0_ang
alpha_rad[i] = alpha_ang[i] / 180.0 * math.pi
theta_ang[i] = theta0_ang
theta_rad[i] = theta_ang[i] / 180.0 * math.pi

def speed_func(u, v, p, cx, Is, p0, P, theta_rad):
	return (Is * 1.0 / (1 - u) - p * (v ** 2) * cx
                * Is * 1.0 / (2 * P * p0 * (1 - u))
                - Is * math.sin(theta_rad) / P)


def high_func(Is, v, theta_rad, P, g):
	return Is * v * math.sin(theta_rad) / (P * g)


def dis_func(Is, v, theta_rad, P, g):
	return Is * v * math.cos(theta_rad) / (P * g)
	

def numerator_func(Is, P, g, v, theta_rad, y, kv1, kv2, kv3, kv4):
	return (Is * 1.0 / (P * g) + (v * high_func(Is, v, theta_rad, P, g)
                - y * (kv1 + 2 * kv2 + 2 * kv3 + kv4) / 6.0) / ((v ** 2)
                * math.sin(theta_rad)))


def arc_func(u, theta_rad, Is, cotq0, v, vt, y, P, g, yt0, kv1, kv2, kv3, kv4):
	return ((vt * 1.0 / yt0) * numerator_func(Is, P, g, v, theta_rad, y
                 , kv1, kv2, kv3, kv4)) / (1 + (cotq0 - (u * Is * vt) / (P
                 * g * yt0)) * (1.0 / math.tan(theta_rad)))


def alpha_func(v, Is, theta_rad, P, p, cya, p0, u, kvv1, kvv2, kvv3, kvv4):
	return ((v * (kvv1 + 2 * kvv2 + 2 * kvv3 + kvv4) / 6.0 + Is
                 * math.cos(theta_rad) / P) / ((p * (v ** 2) * cya * Is)
                 / (2 * P * p0 * (1 - u)) + Is / (1 - u)))

h = 1e-3  #set the step size
EPS = 1e-10
while(y[i] < yt0):
	p = itp.interp1d(high_tab, rho_tab)(y[i])
	c = itp.interp1d(high_tab, speed_tab)(y[i])
	if(alpha_ang[i] + EPS < alpha_tab1[0] or ma + EPS < ma_tab1[0]):
		cx[i] = 0.025
	else:
		cx[i] = (itp.interp2d(alpha_tab1, ma_tab1, cx_tab, kind="linear")
                    (alpha_ang[i], ma))
	cy[i] = (180.0 / math.pi * itp.interp2d(alpha_tab2, ma_tab2, cy_tab
                 , kind="linear")
                (alpha_ang[i], ma))
	ma = v[i] / c

	kv1 = speed_func(u[i], v[i], p, cx[i], Is, p0, P, theta_rad[i])
	kv2 = speed_func(u[i] + h / 2.0, v[i] + h / 2.0 * kv1, p, cx[i]
                         , Is, p0, P, theta_rad[i])
	kv3 = speed_func(u[i] + h / 2.0, v[i] + h / 2.0 * kv2, p, cx[i]
                         , Is, p0, P, theta_rad[i])
	kv4 = speed_func(u[i] + h, v[i] + h * kv3, p, cx[i], Is, p0, P
                         , theta_rad[i])

	v[i+1] = v[i] + h / 6.0 * (kv1 + 2 * kv2 + 2 * kv3 + kv4)
	y[i+1] = y[i] + high_func(Is, v[i], theta_rad[i], P, g) * h
	x[i+1] = x[i] + dis_func(Is, v[i], theta_rad[i], P, g) * h
	
	kvv1 = arc_func(u[i], theta_rad[i], Is, cotq0, v[i], vt, y[i]
                        , P, g, yt0, kv1, kv2, kv3, kv4)
	kvv2 = arc_func(u[i] + h / 2.0, theta_rad[i] + h / 2.0 * kvv1
                        , Is, cotq0, v[i], vt, y[i], P, g, yt0, kv1
                        , kv2, kv3, kv4)
	kvv3 = arc_func(u[i] + h / 2.0, theta_rad[i] + h / 2.0 * kvv2
                        , Is, cotq0, v[i], vt, y[i], P, g, yt0, kv1
                        , kv2, kv3, kv4)
	kvv4 = arc_func(u[i] + h, theta_rad[i] + h * kvv3, Is, cotq0
                        , v[i], vt, y[i], P, g, yt0, kv1, kv2, kv3
                        , kv4)

	theta_rad[i+1] = theta_rad[i] + h / 6.0 * (kvv1 + 2 * kvv2
                         + 2 * kvv3 + kvv4)
	theta_ang[i+1] = theta_rad[i+1] * 180.0 / math.pi
	alpha_rad[i] = alpha_func(v[i], Is, theta_rad[i], P, p, cy[i]
                                  , p0, u[i], kvv1, kvv2, kvv3, kvv4)	
	alpha_ang[i+1] = alpha_rad[i] * 180.0 / math.pi
	u[i+1] = u[i] + h
#       cotq = cotq0 - u[i] * Is * vt / (P * g * yt0)
#       xt[i+1] = cotq * yt0
	i = i + 1
print(i)
print(u[i])
print(alpha_ang[i])


\end{lstlisting}

\end{document}
